%!TEX root = ../crowd_hierarchies_kdd.tex

\section{Estimating the Gain of Queries}
\label{sec:gainestimators}
Previous work~\cite{trushkowsky:2013} has drawn connections between this problem and the species estimation literature~\cite{chao:1992}. However, the proposed techniques therein do not work for queries that specify an exclude list. Moreover, they rely on the presence of a relatively large sample and tend to exhibit negative biases~\cite{hwang:2010, shen:2003}, i.e., they underestimate the expected gain. Negative biases can severely impact entity extraction over large domains since nodes that contain entities that belong in the long tail of the popularity distribution may never be queried as they may be deemed to have zero population. In this section, we first review the existing methodology for estimating the gain of a query. Then we discuss how these estimators can be extended to consider an exclude list. Finally, we propose a new gain estimator for generalized queries $q(k,E)$ that exhibits lower biases, and thus, improved performance, in the presence of little information than previous techniques (see \Cref{sec:exps}).

\subsection{Previous Estimators}
\label{sec:prevest}
Consider a specific node $v \in \hierarchy$. Prior work only considers samples retrieved from the entire population associated with $v$ and does not consider an exclude list. Let $Q$ be the set of all existing samples retrieved by issuing queries against $v$ without an exclude list. These samples can be combined into a single sample corresponding to multi-set of size $n = \sum_{q \in Q} {\sf size}(q)$. Let $f_i$ denote the number of entities that appear $i$ times in this unified sample, and let $f_0$ denote the number of unseen entities from the population under consideration. Finally, let $C$ be the population coverage of the unified sample. i.e., the fraction of the population covered by the sample  $C = \frac{f_1 + f_2 + ..}{f_0 + f_1 + ...}$.

A new query $q(k,\emptyset)$ at node $v$ can be viewed as increasing the size of the unified sample by $k$. Prior work used techniques from species estimation to estimate the expected number of new entities returned in $q(k,\emptyset)$. Shen et al.~\cite{shen:2003}, derive an estimator for the number of new species $\hat{N}_{Shen}$ that would be found in an increased sample of size $k$. The approach assumes that unobserved entities have equal relative popularity. An estimate of the unique elements found in an increased sample of size $k$ is given by:
\begin{equation}
\label{eq:shen}
\hat{N}_{Shen} = f_0\left( 1 - \left(1 - \frac{1 - C}{f_0}\right)^k\right)
\end{equation}
The second term of Shen's formula corresponds to the probability that at least one unseen entity will be present in a query asking for $k$ more entities. Thus, multiplying this quantity with the number of unseen entities $f_0$ corresponds to the expected number of unseen entities present in the result of a new query $q(k,\emptyset)$.

The quantities $f_0$ and $C$ are unknown and thus need to be estimated considering the entities in the running unified sample. The coverage can be estimated by considering the Good-Turing estimator $\hat{C} = 1 - \frac{f_1}{n}$ for the existing retrieved sample. On the other hand, multiple estimators have been proposed for estimating the number of unseen entities $f_0$. Trushkowsky et al.~\cite{trushkowsky:2013} proposed a variation of an estimator introduced by Chao et al.~\cite{chao:1992} to estimate $f_0$. Nevertheless, the authors argue that the original estimator proposed by Chao performs similarly with their approach when estimating the gain of an additional query $q(k,\emptyset)$. Next, we discuss how one can estimate the return of a query $q(k,E)$ in the presence of an exclude list $E$ of size $l$ and potential negative answers.

%The estimator by Chao~\cite{chao:1992} upon which the authors build has been shown to result in considerable negative bias (i.e., tends to underestimate the parameter) in cases where the number of observed entities from a population represents only a {\em small fraction} of the entire population~\cite{hwang:2010}. Notice, that this assumption holds for crowdsourced entity extraction in a large domain, i.e., we are observing only a small portion of the entire population. To address this problem, Hwang and Shen~\cite{hwang:2010} proposed a regression based technique to estimate $f_0$. This estimator is shown to result in significantly smaller bias and empirically outperforms previously proposed estimators, including the one proposed by Chao, when the ratio of retrieved entities to the entire entity population is small. It also performs comparably to previous estimators when the ratio is larger. Notice that the output of this estimator can also be used as a plug-in quantity in \Cref{eq:shen}. However, both the aforementioned estimators are agnostic to an exclude list and potential negative answers. Next, we discuss how one can estimate the return of a query $q(k,l)$ in the presence of an exclude list of size $l$ and potential negative answers.

%\subsection{Queries With an Exclude List}
%\label{sec:excludelist}
%Consider an exclude list of size $l$. As discussed before an exclude list is a set of entities that correspond to invalid worker answers. Considering an exclude list for a query at a node $v \in \hierarchy$ corresponds to limiting our sampling to a restricted subset of the entity population corresponding to node $v$. In fact, we want to estimate the expected return of a query of size $k$ conditioning on the fact that the entities in the exclude list will not be retrieved by any new sample. The latter corresponds to removing these entities from the population under consideration. Thus, the estimates $\hat{f}_0$ and $\hat{C}$ should be updated before applying \Cref{eq:shen} to compute the expected return of a query of size $k$. This can be done by removing the entities included in the exclude list from the running sample for node $v$, recomputing the entity counts $f_i$ and following the techniques presented above for computing the updated estimates for $\hat{f}_0$ and $\hat{C}$. This approach requires that the exclude list is known in advance. To construct an exclude list one can follow a randomized approach, where $l$ of the retrieved entities are included in the list uniformly at random. The generated list can be used to update the frequency counts $f_i$ and estimate the gain of the query. Bootstrapping can also be used to obtain improved estimates.

\subsection{Exclude Lists and Negative Answers}
\label{sec:excludelist}
A query $q(k, E)$ with $E \ne \emptyset$ issued at node $v \in H_D$ effectively limits the sampling to a restricted subset of the entity population corresponding to node $v$. To estimate the expected return of such a query, we need to update the estimates $\hat{f}_0$ and $\hat{C}$ before applying \Cref{eq:shen}, by removing the entities in $E$ from the running sample for node $v$ and updating the frequency counts $f_i$ and sample size $n$. This approach requires that the exclude list is known in advance. We discuss how we construct an exclude list in \Cref{sec:heuristic}.

Next, we study the effect of {\em negative answers} on estimating the gain of future queries. It is possible to issue a query at a specific node $v \in \hierarchy$ and receive no entities, i.e., we receive a negative answer. This is an indication that the underlying entity population of $v$ is empty. In such a scenario, we assign the expected gain of future queries at $v$ and all its descendants to zero. Another type of negative answer corresponds to issuing a query at an ancestor node $u$ of $v$ and receiving no entities for $v$. In this case, we do not update our estimates for node $u$ as entities from other descendants of $u$ may be more popular than entities associated with $u$.

\subsection{Direct Gain Estimation}
\label{sec:newestim}
The techniques reviewed in \Cref{sec:prevest} result in negative bias when the number of observed entities from a population represents only a {\em small fraction} of the entire population~\cite{hwang:2010, shen:2003}. This holds for the large and sparse domains we consider in this paper. To address this problem, Hwang and Shen~\cite{hwang:2010} proposed a regression based technique to estimate $f_0$ and show that it results in smaller biases. However, estimating the total gain of a query requires coupling this new estimator with \Cref{eq:shen}, thus, it may still exhibit negative bias. To eliminate negative bias, we propose a direct estimator for the gain of generalized queries $q(k,E)$ without using \Cref{eq:shen}. We build upon the techniques in~\cite{hwang:2010} and use a regression based technique that captures the structural properties of the expected gain function. \ifpaper The proofs for the results below can be found in the extended version of this paper~\cite{crowdgatherfull}. \fi

Let $S$ denote the total number of entities in the population under consideration and $p_i$ the abundance probability (i.e., popularity) of entity $i$. Given a sample of size $n$ from the population, define $K(n)$ to be $K(n) = \frac{\sum_{i=1}^S (1-p_i)^n}{\sum_{i=1}^S p_i(1-p_i)^{n-1}}$. First, we focus on queries without an exclude list. Later we relax this and discuss queries with exclude lists. We have the following theorem on query gain:

\vspace{-5pt}\begin{theorem}
\label{newgain}
Given a node $v \in \hierarchy$ and a corresponding entity sample of size $n$, let $f_1$ and $f_2$ denote the number of entities that appear exactly once (i.e., singletons) and exactly twice respectively. Let $G$ denote the number of new items retrieved by a query $q(m,\emptyset)$. We have that:

\vspace{-10pt}\begin{equation}
\label{eq:dirgain}
G = \frac{1}{(1 + \frac{K^{\prime}}{n+m})}(K\frac{f_1}{n} - K^{\prime}\frac{f_1(1-\frac{1}{n}2\frac{f_2}{f_1})^m}{n+m})
\end{equation}
where $K = K(n)$ and $K^{\prime} = K(n+m)$.
\end{theorem}
\iftr
\begin{proof}
To derive the new estimator we make used of the generalized jackknife procedure for species richness estimation~\cite{heltshe1983estimating}. Given two (biased) estimators of $S$, say $\hat{S}_1$ and $\hat{S}_2$, let $R$ be the ratio of their biases:
\begin{equation}
R = \frac{E(\hat{S}_1) - S}{E(\hat{S}_2) - S}
\end{equation}
By the generalized jackknife procedure, we can completely eliminate the bias resulting from either $\hat{S}_1$ or $\hat{S}_2$ via
\begin{equation}
S = G(\hat{S}_1, \hat{S}_2) = \frac{\hat{S}_1 - R\hat{S}_2}{1 - R}
\label{eq:jknife}
\end{equation}
provided the ratio of biases $R$ is known. However, $R$ is unknown and needs to be estimated. 

Let $D_n$ denote the number of unique entities in a unified sample of size $n$. We consider the following two biased estimators of $S$: $\hat{S_1} = D_n$ and $\hat{S}_2 = \sum_{j=1}^n D_{n-1}(j)/n = D_n - f_1/n$ where $D_{n-1}(j)$ is the number of species discovered with the $j$th observation removed from the original sample. Replacing these estimators in \Cref{eq:jknife} gives us:
\begin{equation}
S = D_n +\frac{R}{1-R}\frac{f_1}{n}
\end{equation}
Similarly, for a sample of increased size $n+m$ we have:
\begin{equation}
S = D_{n+m} +\frac{R^{\prime}}{1-R^{\prime}}\frac{f^{\prime}_1}{n+m}
\end{equation}
where $R^{\prime}$ is the ratio of the biases and $f^{\prime}_1$ the number of singleton entities for the increased sample. Let $K = \frac{R}{1-R}$ and $K^{\prime} = \frac{R^{\prime}}{1-R^{\prime}}$. Taking the difference of the previous two equations we have:
\begin{equation}
D_{n+m} - D_{n} = K\frac{f_1}{n} - K^{\prime}\frac{f^{\prime}_1}{n+m}
\end{equation}
Therefore, we have:
\begin{equation}
\label{eq:new}
G = K\frac{f_1}{n} - K^{\prime}\frac{f^{\prime}_1}{n+m}
\end{equation}
We need to estimate $K$, $K^{\prime}$ and $f^{\prime}_1$. We start with $f^{\prime}_1$, which denotes the number of singleton entities in the increased sample of size $n+m$. Notice, that $f^{\prime}_1$ is not known since we have not obtained the increased sample yet, so we need to express it in terms of $f_1$, i.e., the number of singletons, in the running sample of size $n$. We have:
\begin{equation}
f^{\prime}_1 = G + f_1 - f_1^c
\end{equation}
where $f_1^c$ denotes the number of old singleton entities from the sample of size $n$ that appeared in the additional query of size $m$. Let $E_1$ denote the set of singleton entities in the old sample of size $n$. We approximate $f_1^c$ by its expected value:
\begin{equation}
\hat{f}_1^c = \sum_{e \in E_1} \Pr[\mbox{e appears in query of size $m$}]
\end{equation}
We compute the probability of an old singleton entity appearing in an additional query as follows. Let $p_e$ denote the popularity of entity $e$. As described before, an additional query of size $m$ corresponds to taking a sample of size $m$ from the underlying entity population without replacement. However, $m$ is significantly smaller compared to the size of the underlying population, thus, we can consider a that taking a sample of size $m$ corresponds to taking a sample {\em with replacement}. Following this we have that:
\begin{equation}
\Pr[\mbox{e appears in query of size $m$}] = 1 - (1-p_e)^m
\end{equation}
Following a standard approach in the species estimation literature we assume that the popularity of retrieving a singleton entity again is the same for all singleton entities. This popularity can be computed using the corresponding Good-Turing estimator considering the running sample. We have:
\begin{equation}
\forall e \in E_1, p_e = p_1 = \hat{\theta}(1) = \frac{1}{n}2\frac{f_2}{f_1}
\end{equation}
where $f_2$ is the number of entities that appear twice in the sample and $f_1$ is the number of singletons. 
Eventually we have that:
\begin{equation}
\hat{f}_1^c = f_1(1 - (1-p_1)^m)
\end{equation}
and
\begin{equation}
f^{\prime}_1 = G + f_1(1-p_1)^m
\end{equation}
Replacing the last equation in \Cref{eq:new} we have:
\begin{align}
&G = K\frac{f_1}{n} - K^{\prime}\frac{G + f_1(1-p_1)^m}{n+m} \nonumber \\
&G = K\frac{f_1}{n} - K^{\prime}\frac{G}{n+m} - K^{\prime}\frac{f_1(1- P)}{n+m} \nonumber \\
&G(1 + \frac{K^{\prime}}{n+m}) = K\frac{f_1}{n} - K^{\prime}\frac{f_1(1- P)}{n+m} \nonumber \\
&G = \frac{1}{(1 + \frac{K^{\prime}}{n+m})}(K\frac{f_1}{n} - K^{\prime}\frac{f_1(1-p_1)^m}{n+m}) \nonumber
\end{align}
\end{proof}
\fi
All quantities apart from $K$ and $K^{\prime}$ in \Cref{eq:dirgain} are known. The value of $K$ can be estimated using the regression approach introduced by Hwang and Shen~\cite{hwang:2010}. \iftr From the Cauchy-Schwarz inequality we have that:
\begin{equation}
K = \frac{\sum_{i=1}^S (1-p_i)^n}{\sum_{i=1}^S p_i(1-p_i)^{n-1}} \geq \frac{(n-1)f_1}{2f_2}
\end{equation}
This can be generalized to:
\begin{equation}
K=\frac{nf_0}{f_1} \geq \frac{(n-1)f_1}{2f_2} \geq \frac{(n-2)f_2}{3f_3} \geq \dots
\end{equation}
Let $g(i) = \frac{(n-i)f_i}{(i+1)f_{i+1}}$. From the above we have that the function $g(x)$ is a smooth monotone function for all $x \geq 0$. Moreover, let $y_i$ denote a realization of $g(i)$ mixed with a random error. Hwang and Shen how one can use an exponential regression model to estimate $K$. The proposed model corresponds to:
\begin{equation}
y_i = \beta_0\exp(\beta_1i^{\beta_2}) + \epsilon_i
\end{equation}
where $i = 1, \dots, n-1$, $\beta_0 > 0$, $\beta_1 < 0$, $\beta_2 >0$ and $\epsilon_i$ denotes random errors. It follows that $K = \beta_0$. \fi
To estimate the value of $K^{\prime}$ for an increased sample of size $n+m$, we first show that $K$ increases monotonically as the size of the running sample increases. 

\begin{lemma}
\label{monotonicity}
The function $K(n) = \frac{\sum_{i=1}^S (1-p_i)^n}{\sum_{i=1}^S p_i(1-p_i)^{n-1}}$ increases monotonically, i.e., $K(n+m) \geq K(n), \forall n,m > 0$.
\end{lemma}
\iftr
\begin{proof}
In the remainder of the proof we will denote $K(n+m)$ as $K^{\prime}$. By definition we have that $K = \frac{\sum_{i=1}^S (1-p_i)^n}{\sum_{i=1}^S p_i(1-p_i)^{n-1}}$ and $K^{\prime} = \frac{\sum_{i=1}^S (1-p_i)^{n+m}}{\sum_{i=1}^S p_i(1-p_i)^{n+m-1}}$. We want to show that:

{\small
\begin{align}
&\frac{\sum_{i=1}^S (1-p_i)^{n+m}}{\sum_{i=1}^S p_i(1-p_i)^{n+m-1}} \geq \frac{\sum_{i=1}^S (1-p_i)^n}{\sum_{i=1}^S p_i(1-p_i)^{n-1}} \nonumber \\
&\sum_{i=1}^S (1-p_i)^{n+m}\sum_{j=1}^S p_j(1-p_j)^{n-1} \geq \sum_{i=1}^S p_i(1-p_i)^{n+m-1}\sum_{j=1}^S (1-p_j)^n\nonumber \\
&\sum_{i,j:i\prec j}[(1-p_i)^{n+m}p_j(1-p_j)^{n-1} - p_i(1-p_i)^{n+m-1}(1-p_j)^n + \nonumber \\
& + (1-p_j)^{n+m}p_i(1-p_i)^{n-1} - p_j(1-p_j)^{n+m-1}(1-p_i)^n] \geq 0 \nonumber \\
%&\sum_{i,j:i\prec j}[(1-p_i)^{n}(1-p_j)^{n-1}p_j((1-p_i)^{m} - (1-p_j)^{m})  + \nonumber \\
%& - (1-p_j)^{n}p_i(1-p_i)^{n-1}((1-p_i)^{m} - (1-p_j)^{m}) \geq 0 \nonumber \\
&\sum_{i,j:i\prec j}[(1-p_i)^{n-1}(1-p_j)^{n-1}(p_j-p_i)((1-p_i)^{m} - (1-p_j)^{m}) \geq 0
\end{align}}

But the last inequality always holds since each term of the summation is positive. In particular, if $p_j \geq p_i$ then
also $1-p_i \geq 1-p_j$ and if $p_j \leq p_i$ then $1-p_i \leq 1-p_j$.
\end{proof}
\fi
Given the monotonicity of function $K$, we model $K$ as a generalized logistic function of the form $K(x) = \frac{A}{1+exp(-G(x-D))}$. As we observe samples of different sizes for different queries we estimate $K$ as described above and therefore we observe different realizations of $f(\cdot)$. Thus, we can learn the parameters of $f$ and use it to estimate $K^{\prime}$. In the presence of an exclude list of size $l$ we follow the approach described in \Cref{sec:excludelist} to update the quantities $f_i$ and $n$ used in the analysis above. 

%\subsection{Negative Answers and Gain Estimation}
%\label{sec:neg}
%Next, we study the effect of {\em negative answers} on estimating the gain of future queries. It is possible to issue a query at a specific node $v \in \hierarchy$ and receive no entities, i.e., we receive a negative answer. This is an indication that the underlying entity population of $v$ is empty. In such a scenario, we assign the expected gain of future queries at $v$ and all its descendants to zero. 
%
%Another type of negative answer corresponds to the scenario where we issue a query at an ancestor node $u$ of $v$ and receive no entities associated with $v$ but received some entities for $u$. Notice, that in this case, we do not have enough information to update our estimates for node $u$. The reason is that due to the restricted query size entities from other descendants of $u$ may be more popular with respect to the popularity distribution of $u$.